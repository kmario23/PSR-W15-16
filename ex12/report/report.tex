\documentclass[a4paper]{article}
\usepackage{a4wide}
\usepackage{amsmath}
\usepackage{amsfonts}
  \DeclareMathOperator*{\argmax}{arg\,max}
  \newcommand{\ex}[1]{{\mathbb E}\left[ #1 \right]}
  \newcommand\norm[1]{\left\lVert#1\right\rVert}
\usepackage{booktabs}
\usepackage{csquotes}
\usepackage{upquote}
\usepackage{float}
\usepackage{graphicx}
\usepackage{enumerate}
\usepackage{subcaption}
\usepackage[most]{tcolorbox}
\usepackage{xcolor}
\usepackage{varwidth} 


\title{Pattern and Speech Recognition WS2015-16 \\ Exercise 9}
\author{Atanas Poibrenski(2554135), Marimuthu Kalimuthu(2557695), Furkat Kochkarov(2557017)}

\begin{document}
	\tcbset{
		enhanced,
		colback=red!5!white,
		boxrule=0.1pt,
		colframe=red!75!black,
		fonttitle=\bfseries
	}

\maketitle 
\begin{center}
	\textbf{Hidden Markov Model}
\end{center}

\section*{Exercise-2}
\begin{itemize}
	\item Done.
\end{itemize}

\section*{Exercise-3}
	\begin{itemize}
		\item The results are:
		
			\begin{tcolorbox}
				Accuracy = 91.9\% \\
				Precision = 75.87\% \\
				Recall = 48.46\% \\
				F-Score = 59.14\% \\
			\end{tcolorbox}						    
	\end{itemize}


\section*{Exercise-4}
\begin{itemize}
	\item x = 2, 1, 5, 6, 2, 1, 3, 1, 6, 2 \newline
	  $\pi$ = F, F, L, F, F, F, L, F, L, F \newline
	  
	  The maximum likelihood probability of transitioning from state \textbf{i} to state \textbf{j} is just the number of times we transition from \textbf{i} to \textbf{j} divided by the total number of times we are in state \textbf{i}. In other words, the maximum likelihood parameter corresponds to the fraction of the time when we were in state \textbf{i} that we transitioned to \textbf{j} \newline
	  
	  Formula to caculate transition matrix: \newline
	  
	  $\hat{A}_{ij} = \frac{\sum_{t=1}^{T} 1 \{z_{t-1} = s_{i} \wedge z_{t} = s_{j}\} }{\sum_{t=1}^{T}1 \{z_{t-1} = s_{i}\}}$
	  
	  where \textbf{T} is the sequence length and we transition from state \textbf{i} to \textbf{j}.
	
	\begin{tcolorbox}
		$\bordermatrix{
			  &F     &L    \cr
			F &3/6   &3/6  \cr
			L &3/3   &0    \cr			
			}$
	\end{tcolorbox}
	
	The emission probabilities for state \textbf{j} are calculated as \newline
	(\# times state \textbf{j} emitted symbol \textbf{s}) $/$ (\# times state \textbf{j} occurred).
	
	\begin{tcolorbox}
		$\bordermatrix{
			  &1     &2   &3   &4   &5   &6    \cr
			F &3/7   &3/7 &0   &0   &0   &1/7  \cr
			L &0     &0   &1/3 &0   &1/3 &1/3  \cr			
		}$
	\end{tcolorbox}	    	
\end{itemize}

\end{document}